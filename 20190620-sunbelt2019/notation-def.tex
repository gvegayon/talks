% Mathematical functions
\newcommand{\isone}[1]{{\boldsymbol{1}\left( #1 \right)}}
\renewcommand{\Pr}[1]{{\mathbb{P}\left(#1\right) }}
\newcommand{\f}[1]{{f\left(#1\right) }}
\newcommand{\Prcond}[2]{{\mbox{Pr}\left(#1\vphantom{#2}\;\right|\left.\vphantom{#1}#2\right)}}
\newcommand{\fcond}[2]{{f\left(#1|#2\right) }}
\newcommand{\Expected}[1]{{\mathbb{E}\left\{#1\right\}}}
\newcommand{\ExpectedCond}[2]{{\mathbb{E}\left\{#1\vphantom{#2}\;\right|\left.\vphantom{#1}#2\right\}}}

\newcommand{\Likelihood}[2]{\text{L}\left(#1 \left|\vphantom{#1}#2\right.\right)}
\newcommand{\sufstats}[1]{s\left(#1\right)}
\renewcommand{\exp}[1]{\mbox{exp}\left\{#1\right\}}
\newcommand{\transpose}[1]{{#1}^\mathbf{t}}

% Objects
\newcommand{\params}{\theta}
\newcommand{\Params}{\Theta}
\newcommand{\Graph}{\mathbf{G}}
\newcommand{\graph}{\mathbf{g}}
\newcommand{\GRAPH}{\mathcal{G}}
\newcommand{\Adjmat}{\mathbf{A}}
\newcommand{\adjmat}{\mathbf{a}}
\newcommand{\DEPVAR}{\mathcal{Y}}
\newcommand{\Depvar}{Y}
\newcommand{\depvar}{y}

\newcommand{\conf}{\mathbf{z}}
\newcommand{\Conf}{\mathbf{Z}}
\newcommand{\CONF}{\matchal{Z}}

\newcommand{\INDEPVAR}{\mathcal{X}}
\newcommand{\Indepvar}{\mathbf{X}}
\newcommand{\indepvar}{\mathbf{x}}


% tricks for two column in Rmarkdown
\def\begincols{\begin{columns}[c]}
\def\begincol{\begin{column}[c]}
\def\endcol{\end{column}}
\def\endcols{\end{columns}}

\graphicspath{{./fig/}{.}}

% Theme 
\usecolortheme{seagull}
\usefonttheme{structurebold}

\setbeamertemplate{footline}[frame number]

\addtobeamertemplate{background canvas}{\transfade[duration=.5]}{}

%% NEED THIS FOR CANCY TEX
\usepackage{pstricks}

% Colors
\definecolor{USCCardinal}{HTML}{990000} % 153 0 0 in RGB
\definecolor{USCGold}{HTML}{FFCC00}
\definecolor{USCGray}{HTML}{CCCCCC}

\newcommand{\uscred}[1]{\color{USCCardinal}\textbf{#1}\color{black}{}}
\newcommand{\uscgold}[1]{\color{USCGold}\textbf{#1}\color{black}{}}

% To use the function \sout
\usepackage{ulem}
\usepackage{tabularx, booktabs}

% \bibliography{bibliography.bib}


% tricks for two column
\def\begincols{\begin{columns}[c]}
\def\begincol{\begin{column}[c]}
\def\endcol{\end{column}}
\def\endcols{\end{columns}}

\usepackage{tabularx}

\usepackage{tikz}

\newcommand{\includetikz}[2]{
\begin{figure}
\scalebox{#2}{
\input{#1}
}
\end{figure}
}

% \usefonttheme{professionalfonts}
% \usefonttheme{serif}
% \usepackage{fontspec}
% \setmainfont{TeX Gyre Heros}
% \setbeamerfont{note page}{family*=pplx,size=\footnotesize}

\usecolortheme{seagull}
\usefonttheme{structurebold}

\setbeamertemplate{footline}[frame number]

\addtobeamertemplate{background canvas}{\transfade[duration=.5]}{}

% Some neat trick that allows me to use title